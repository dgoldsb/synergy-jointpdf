% Version: 0.0

\documentclass[../main.tex]{subfiles}

\begin{document}

\section{Methods}
\label{appendix_methods}

\subsection{Alternative deciding conditions}
\label{alternative_deciding}

We support four deciding conditions, each of which fit with a different design choice in simplifying reality.
The default is the \texttt{totaleffect} condition.
We offer these alternatives as the choice of condition can effect which transition tables can be contructed using networks.
These conditions are:
%
\begin{itemize}
\itemsep0em 
\item (\texttt{totaleffect}) Each function affecting a gene outputs a \textit{change} in the target gene, multiple effects are added. The expression level remains capped ($0 \le m \le l$). This implies that one strong inhibitor can overpower several weak stimulators completely.
\item (\texttt{average}) Each function affecting a gene outputs a \textit{suggested value} of the target genes, multiple effects combined using a weighted average and rounded to the nearest integer value. This implies that when a gene is both stimulated and inhibited, it is much more likely to end up in a semi-activated state, and not either in a deactivated state or an activated state.
\item (\texttt{down}) Each function affecting a gene outputs a \textit{suggested value} of the target genes, the lowest is selected. This method assumes inhibition is dominant.
\item (\texttt{majority}) Each function affecting a gene outputs a \textit{suggested value} of the target genes, the most commenly suggested value is chosen. In case of a tie a random value is selected.
\end{itemize}

\subsection{Searching for motifs}

As a validation method, we implemented a search function to explore random samples.
This allows us to search sets of randomly generated motifs for a particular motif of our choosing, for instance one that is common in nature.
This function takes a network-type motif as an input, as well as a set of motifs of any kind.
First, it converts this input motif to a transition table.
Then, all possible variations of the transition table that represent the same motif are produced.
After all, if we switch the labels of two genes in the motif the transition table changes, while the network-motif stays the same.
For each transition table in the sample, we check if it is in the set of transition tables we produced in the previous step.

%TODO meer hier?

\end{document}