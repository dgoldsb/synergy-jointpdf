% Version: 1.1

\documentclass[../main.tex]{subfiles}

\begin{document}
% Kern thesis: ik heb twee modellen, een supernaif en een een beetje naief, ik kijk of het verschil in eigebschappen de resilience en memory verklaard, en of dit verschil ook synergyverschil oplevert
%TODO refer to hypotheses by number

% Summary and restate thesis statement
% Answer (1) 
In this study we investigated the link between the synergy of a system and its resilience and memory.
We found that there is a positive correlation between synergy and resilience, which implies that a synergistic control to improve resilience is plausible in gene regulation.
It was also found that there is only a weak negative correlation between synergy and memory, implying that synergy could be a mechanism to improve resilience while retaining memory.
The observed relation between synergy and nudge impact is in line with the findings of Quax et al.~\ref{quax2017quantifying}.s
However, we also found that the amount of memory in a system is a much greater force on the resilience than the synergy is, and the dominant force in random gene regulation systems.
This is not unsurprising; a static system is per definition very sensitive to nudges, and a system where there is no link between now and the future is not disturbed by nudges.
Furthermore, we found that neither the combination of high memory and low nudge impact nor the combination of low memory and high nudge impact does occur.
% STRESS THAT CORRELATIONS PROBABLY DIFFER DUE TO DIFFERENT RANGES

% Draw a broad conclusion
This suggests that synergy could be a form of control on the resilience of a system, but that this control is not present in biological networks at low level due to the lack of synergistic structures with few components in nature.
We ignored any synergy at a higher level, however, as well as any selective pressure in our model.
As such, we suggest that this synergistic control might be emergent in natural networks at a higher level; not at the level of GRN motifs, but at the level of combinations of GRN motifs.
This is an interesting question to tackle in future research, and due computational limitations not one that can be answered using our methodology.

% Answer (2) means
We compared two types of random systems with discrete states and time steps.
One system is completely random in nature, and is evolved following a completely random state transisition table.
The other is meant to mimic GRNs, and is evolved following a state transition table computed from a random GRN.
While both groups are random in nature, the latter was found to sample from a small subspace in the sample space of the completely random tables.
We found that the group of GRN-like systems score worse in both synergy and memory.
This is opposite to our hypothesis, as we predicted that biological systems would score high in both these properties.
Our understanding is that synergetic relations are difficult to build using the components of gene regulation networks; gene regulation works mostly using positive and negative feedback, with an occasional AND-gate.
Typical structures that provide synergy, such as a XOR-gate, are unlikely to come into existence from randomly combining these components.
The difference in memory is more to attribute to an underestimation of the amount of memory in completely random systems.
The amount of memory in gene regulation systems is not low by any means, preserving more than 50\% of the mutual information typically.
However, the amount of mutual information preserved in random networks is usually exceptionally high, close to 90\%.

We found that gene regulation systems where more resilient to nudges than random systems.
This is in line with our hypothesis, as biological networks have a strong need for resilience to function.
However, contrary to our hypotheses, this difference in resilience was observed regardless of how many variables in the system where nudged.
The difference in resilience can likely be attributed to the great difference in memory, as a higher memory was found to correlate with a decreased resilience.
We suggest that the type of relations in gene regulatory networks cause the system to only evolve to a select few states.
As this results in multiple states all leading to the same state in the transitition table, information is lost and the memory will be lower.
We see that this lower memory is the dominant force on the resilience, as we observe an on average lower nudge impact in biological networks.

These results where found to be consistent with an increasing network size, as well with an higher 'resolution' by increasing the number of possible states of each variable.
Larger networks and those with a higher resolution yielded more significant results.
These larger systems are also closer to reality, as GRNs often include tens to hundreds of genes.
In theory a higher resolution is also closer to reality, although this does depend on whether our extension of the conversion model from GRN to transition table is correct for multi-valued logic systems.
% Good sign that results are consistent amongst expression levels.

% Variance (addition to means)
We observed that the GRN-like systems occupied a much wider range in synergy, memory, and resilience.
Furthermore, this discrepancy in their ranges becomes greater when increasing the system size or number of possible states.
This is likely due to the increase in the sample space; the sample space grows strongly when increasing either the system size or number of possible states.
If we are correct in our observation that the biological section of this sample space is small and has unique properties, then an increase in the sample space will make it less and less likely that random samples are drawn from the biological section.
This could explain why the random measurements get bunched up around a most common point in the 3D space, and are not found anymore in the area of space occupied by the biological tables.

% Answer (3)
We could not draw a conclusion on a difference in the impact of a single-target nudge versus a multi-target nudge with our current methodology.
However, we did find significant results with the same sign in our experiments, which suggest that if there is a difference it is a matter of magnitude, not whether there is an effect at all.
% Answer (4)
Random systems where found to be well-balanced between synergy and redundancy at all levels, whereas gene regulation systems are more prone to contain excess redundancy or synergy at some scale.

\end{document}
