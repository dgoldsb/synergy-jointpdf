% Version: 0.0

\documentclass[../main.tex]{subfiles}

\begin{document}

% Summary
Kern thesis: io heb twee modelleb, een supernaif en een een beetje naief, ik kihk of het verschil in eigebschappen de resilience en memory verklaard, en of sit verschil ook synergygerschio oplveret

% discussiëren on: in fact I am seeing if the restraint to go fron network to trans table explaibs hiigher resiliencenand memory, and if it is paired with synergy. 4 possible results: if synergy does not match the resukt it is not part of resilienvr and memory, id it does either positive result or negative with ansuggestion for further research: optimizing bron i networks

% Means
Largest networks are closest to reality, so most important.
Same with the most expression level, but the model might be more shaky here.
Good sign that results are consistent amongst expression levels.

% Variance
This is likely due to the increase in the sample space; the sample space grows strongly when increasing either the system size or number of possible states.
If we are correct in our observation that the biological section of this sample space is small and has unique properties, then an increase in the sample space will make it less and less likely that random samples are drawn from the biological section.
This could explain why the random measurements get bunched up around a most common point in the 3D space, and are not found anymore in the area of space occupied by the biological tables.

% Spearman
Why the triangle

No correlation memory and resilience is kind of weird, you expect extremely static systems to have low resilience, and vice versa

\end{document}
