% Version: 0.0

\documentclass[../main.tex]{subfiles}

\begin{document}

% Means

% Variance
This is likely due to the increase in the sample space; the sample space grows strongly when increasing either the system size or number of possible states.
If we are correct in our observation that the biological section of this sample space is small and has unique properties, then an increase in the sample space will make it less and less likely that random samples are drawn from the biological section.
This could explain why the random measurements get bunched up around a most common point in the 3D space, and are not found anymore in the area of space occupied by the biological tables.

% Spearman
Why the triangle

\end{document}
