% Version: 1.1

\documentclass[../main.tex]{subfiles}

\begin{document}
% Kern thesis: ik heb twee modellen, een supernaif en een een beetje naief, ik kijk of het verschil in eigebschappen de resilience en memory verklaard, en of dit verschil ook synergyverschil oplevert
%TODO refer to hypotheses by number
%TODO iets korter
%TODO Avg. correlatie bespreken (verbeteren stukje)


% Summary and restate thesis statement
% Answer (1) 
In this study we investigated the link between the synergy and memory of a system and its resilience.
We found a strong positive correlation between the system memory and the impact of nudges.
This is not unsurprising; a static system is per definition very sensitive to nudges, and a system where there is no link between now and the future is not disturbed by nudges.
We also observed a negative correlation between the synergy and the nudge impact.
This observed relation is in line with the findings of Quax et al.~\ref{quax2017quantifying}.
However, we found that synergy is uncorrelated with the nudge impact when we normalize for the system memory.
Memory does remain correlated with the nudge impact when we normalize for the system memory.
We did observe a negative correlation between the system synergy and the system memory.
These results were the same for random and biologically possible motifs.

As such, we conclude for hypotheses (1) and (2) that there is no direct correlation between synergy and nudge resilience.
There does appear to be an indirect correlation memory, as there is a correlation that disappears when we normalize for memory.
We did observe a correlation between synergy and memory, but this relationship was of the opposite sign as hypothesized in hypotheses (3) and (4).
A summary of the correlation findings is shown in Fig.~\ref{venn_results}.

% Now we can draw the sets:
\def\firstcircle{(0:-0.9cm) circle (2cm)}
\def\thirdcircle{(0:0.9cm) circle (2cm)}
\begin{figure}[ht]
\begin{center}
\begin{tikzpicture}
    \draw \firstcircle;
    \draw \thirdcircle;
    
    \begin{scope}[fill opacity=0.5]
        \clip \firstcircle;
        \fill[orange] \thirdcircle;
    \end{scope}
    
    \begin{scope}[even odd rule, fill opacity=0.5]
        \clip \thirdcircle (-4,-2) rectangle (2,2);
        \fill[yellow] \firstcircle;
    \end{scope}
    
    \begin{scope}[even odd rule, fill opacity=0.0]
        \clip \firstcircle (-2,-2) rectangle (2,2);
        \fill[red] \thirdcircle;
    \end{scope}
    
    \begin{scope}[even odd rule, fill opacity=0.3]
        \clip \firstcircle (-2,-2) rectangle (2,2);
        \clip \thirdcircle (-2,-2) rectangle (2,2);
    \end{scope}
    
    \node (x) at (-2,0)  {$\rho > 0 $};
    \node (y) at (2,0)   {$\rho \approx 0$};
    \node (r) at (0,0)   {$\rho > 0$};
    \node (s) at (-3,2.3) {$\rho(\mathrm{memory}, \mathrm{resilience})$};
    \node (w) at (3,2.3) {$\rho(\mathrm{synergy}, \mathrm{resilience})$};
    
\end{tikzpicture}
\end{center}
\caption{Suggested correlations between memory and resilience, and synergy and resilience}
\label{venn_results}
\end{figure}

% Answer (2) means
%TODO memory is so high because chance is small to get the same state multiple times
%TODO maar gecorrigeerd voor memory, dus als de memory gelijk is in de twee (ofwel p(synergy | memory)), is dan de synergy hoger in de GRNs?
We compared two types of random systems with discrete states and time steps.
One system is completely random in nature, and is evolved following a completely random state transisition table.
The other is meant to mimic GRNs, and is evolved following a state transition table computed from a random GRN. %TODO of op zijn minst: laat deze set een strakkere maar nog steeds bredere set zijn rondom de 'echte' GRNs?
While both groups are random in nature, the latter was found to sample from a small subspace in the sample space of the completely random tables.
We found that the group of GRN-like systems score worse in both synergy and memory.
This is opposite to our hypothesis, as we predicted that biological systems would score high in both these properties.
Our understanding is that synergetic relations are difficult to build using the components of gene regulation networks; gene regulation works mostly using positive and negative feedback, with an occasional AND-gate.
Typical structures that provide synergy, such as a XOR-gate, are unlikely to come into existence from randomly combining these components.
The difference in memory is more attributable to an underestimation of the amount of memory in completely random systems.
The amount of memory in gene regulation systems is not low by any means, preserving more than 50\% of the mutual information typically.
However, the amount of mutual information preserved in random networks is usually exceptionally high, close to 90\%.

%TODO again, normalize for memory? what then?
We found that gene regulation systems where more resilient to nudges than random systems.
This is in line with our hypothesis, as biological networks have a strong need for resilience to function.
However, contrary to our hypotheses, this difference in resilience was observed regardless of how many variables in the system where nudged.
The difference in resilience can likely be attributed to the great difference in memory, as a higher memory was found to correlate with a decreased resilience.
We suggest that the type of relations in gene regulatory networks cause the system to only evolve to a select few states.
As this results in multiple states all leading to the same state in the transitition table, information is lost and the memory will be lower.
We see that this lower memory is the dominant force on the resilience, as we observe an on average lower nudge impact in biological networks.

These results where found to be consistent with an increasing network size, as well with an higher 'resolution' by increasing the number of possible states of each variable.
Larger networks and those with a higher resolution yielded more significant results.
These larger systems are also closer to reality, as GRNs often include tens to hundreds of genes.
In theory a higher resolution is also closer to reality, although this does depend on whether our extension of the conversion model from GRN to transition table is correct for multi-valued logic systems.
% Good sign that results are consistent amongst expression levels.

% Variance (addition to means)
% RICK: ik snap deze uitleg niet helemaal... zou het ook niet kunnen zijn dat de natuur misschien deze manier van GRNs modelleren gebruikt (dus edges en 2-to-1 edges) om juist te zorgen dat het gemakkelijk een brede varieteit aan memory en synergy kan bereiken, wat moeilijk zou zijn door de evolueren in de random systems? DYLAN: ik heb het duidelijker geschreven, maar er is geen evolutie van random systemen; random is echt random, en elke transition table heeft een gelijke kans getrokken te worden
We observed that our biologically possible sampled systems had a much higher variance in their synergy, memory, and resilience.
This is odd, as all these samples can also be sampled from the set of random systems.
Furthermore, this discrepancy in their ranges becomes greater when increasing the size of the sample space, either through increasing $l$ or $n$.
This can be explained through this increase in the sample space.
If we hypothesize that the subspace of biologically possible systems grows less quickly than the overall sample space and has properties much different from the rest of the sample space, then an increase in the sample space will make it less and less likely that random samples are drawn from the biological section.
This could explain why the random measurements get bunched up around a most common point in the 3D space, and are not found anymore in the area of space occupied by the biological tables.

% Answer (3)
We could not draw a conclusion on a difference in the impact of a single-target nudge versus a multi-target nudge with our current methodology.
However, we did find significant results with the same sign in our experiments, which suggest that if there is a difference it is a matter of magnitude, not whether there is an effect at all.
% Answer (4)
Random systems where found to be well-balanced between synergy and redundancy at all levels, whereas gene regulation systems are more prone to contain excess redundancy or synergy at some scale.

%TODO misschien wat diepere, bredere conclusie aan het eind die alles combineert tot iets van een inzicht? mogelijke implicaties voor een en ander?
This suggests that synergy does not function as a mechanism to improve resilience.
Biologically possible tables is a very small space
GRN systems are more extrem in synergy and redundancy.
We found both single and multi nudges had similarly directed effects.

\end{document}
