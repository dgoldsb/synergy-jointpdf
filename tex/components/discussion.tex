% Version: 0.0

\documentclass[../main.tex]{subfiles}

\begin{document}

Ideally, we would use a sample of empirically obtained datasets.
However, due to the limited availability of datasets that are both suitable (i.e. an ODE model) and small enough for analysis, this was not possible.
Our resort to generated GRN-like networks was the next best alternative, but raises doubts about the extend to which our results can be generalized to all biological complex systems.
The use of correlations to replace the rest of the network makes it so that the framework is not suited for prolonged simulations.

As the accurate description of GRNs as ODE systems is relatively new territory, networks are not very accurate yet.
The ODE system used in this study has a corresponence of nearly 50\% to emperical data.
We think that, while these models are still not truly representative for real systems, these sytems should represent the type of dynamics in GRNs fairly well.
Much research is done on Boolean models, which simply work with a binary activation/deactivation system.
With the addition of reaction rates, the model used captures reality better than these Boolean models.
Furthermore, we think that the addition of reaction rates, even when not fully corresponding to reality, will add enough extra informaton to bridge the problem that network motifs only do not predict function.

% TODO: too slow for 5-sized motifs
% TODO: experimental setting 0 really does not work, even without using Rick's synergy (but with his nudge)

% If I have more time...
% \item An actual GRN motif is optimized for memory and resilience

% If I have even more time...
% \item An actual GRN is at the Pareto boundary of the memory/resilience cost function
% \item Synergy is found at a low level in biological networks, the level of common network motifs 
% \item Synergy is found at a low level in trained random GRNs
% \item The (DJ graph) indicates a level of synergistic control that is greater than random

\end{document}
