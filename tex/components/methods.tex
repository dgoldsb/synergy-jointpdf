% Version: 0.0

\documentclass[../main.tex]{subfiles}

\begin{document}

\subsection{Dataset}

We use an ODE-model for the sea urchin endomesoderm GRN, which is responsible for early development of the embryo \cite{kuhn2009monte, davidsonlab}.
This GRN is well-studied, and manageable in size with a total of around 60 genes, depending on the model version.
The reaction rate parameters were fitted using a Monte Carlo method, and the resulting system has a 48\% correspondence with empirical data \cite{kuhn2009monte}.
The model contains of 54 genes, 140 variable species, 278 reactions and 287 parameters.
The model is accessed in the form of an SBML-file, provided by the authors.

\subsection{Complexity Profile}

% Introduction to our synergy profile
% Link to introduction
In previous research, the mutual information within a system has been used to investigate complexity in systems of independent and dependent random variables.
A multiple mutual information-based profile has been proposed in the literature that is able to give insights beyond pairwise relations.
A synergy profile can be considered as a plot of the fraction captured of the total mutual information between all input variables and all output variables versus the number of variables taken into consideration, or
%
\begin{equation}
C_\mathrm{mult}(k) = \frac{1}{\binom{n}{k}}\frac{\sum_{X_i \in [X]^k} [\mathrm{I}(X_i;Y)]}{\mathrm{I}(X;Y)}
\end{equation}

This profile has the property $C_\mathrm{mult}(0) = 0$, as for $k = 0$ there is only the empty set, which has zero mutual information with $Y$.
In addition, we know that $C_\mathrm{mult}(n) = 1$, as this simply results in
%
\begin{align}
C_\mathrm{mult}(k) 
&= \frac{1}{\binom{n}{k}}\frac{\sum_{X_i \in [X]^k} [\mathrm{I}(X_i;Y)]}{\mathrm{I}(X;Y)} \\
&= \frac{\mathrm{I}(X;Y)}{\mathrm{I}(X;Y)} \\
&= 1
\end{align}

Finally, we can show that this profile is non-decreasing.
We can proof this by imagining an extreme case, where out of variable set $Z$ only $z_1$ provides direct information about the output variable, whereas the rest only provide information when all considered together.
When considering a subset size $1 \le k < n_z $, the complexity will be
%
\begin{equation}
C_\mathrm{mult}(k) = \frac{1}{\binom{k}{n_z}} \mathrm{I}(z_1;Y)
\end{equation}
%
where $n_z$ is the size of set $Z$.
As the mutual information term is constant for $k$, only the fraction determines the complexity value.
As with increasing $k$ fewer and fewer subsets can be made, the function will always be increasing or stagnant, the latter of which is possible only if $C_\mathrm{mult}(0) = 0$.

As there are many ways to take subsets when $1 < k < n$, we average the sum over all subsets.
However, it is also possible to rewrite this to
%
\begin{equation}
C_\mathrm{mult}(k) = \frac{\max_{X_i \in [X]^k} [\mathrm{I}(X_i;Y)]}{\mathrm{I}(X;Y)}
\end{equation}
%
to focus on extreme values within the set of subsets of size $k$.

% Go into the math behind it, using the NPEET library documentation
We apply this complexity measure on continuous distributions, forcing us to use approximation methods for the mutual information term.
we use the k-Nearest Neighbor method described by Kraskov, and implemented in Python by Ver Steeg et al. \cite{kraskov2004estimating, versteeg2013NPEET}.
For this method, a set of $N$ of samples is drawn from the underlying distributions of both the input and the output system.
This results in two matrices: an input sample matrix of size $N \times X$, and an output sample matrix of size $N \times Y$, filled with continuous values.
In both matrices, the $N$-size dimension represents samples, and the $X$- or $Y$-sized dimension represents the variables of the respective system.
These samples are used in the NPEET library for a multiple mutual information approximation.
Specifically, Kraskov's first algorithm is used, with the addition of the division by a logarithm to report answers in bits.
This is
%
\begin{equation}
\mathrm{I}_\mathrm{kNN}(X;Y) = \frac{\psi(k) - < \psi(n_x + 1) + \psi(n_y + 1) > + \psi(N)}{\log(2)}
\end{equation}
%
where $\psi(x)$ is the digamma function, and $n_x(i)$ and $n_y(i)$ respectively are the number of points within $\epsilon(i)/2$ of $x_i$ an $y_i$. Here, $\epsilon(i)/2$ is the distance to the $k$-th nearest neighbor of the point in $(X,Y)$-space.

% Subsets
We now can approximate the total mutual information between the input- and the output system.
To produce a plot, we must obtain a mutual information estimate for subsets of each size $1 \le k \le X$, versus the entire output system.
Too obtain this overall estimate for each subset size, we first find the set of mutual informations between every possible subset of $X$ of size $k$, versus the entire system $Y$.
Then, we take the average of the set and divide by the total mutual information between the input- and output system to arrive at the value that corresponds to subset size $k$ in our synergy profile.
We repeat this system for each possible subset size.
Finally, the plot is produced by plotting the $k$ against the corresponding measure.

\subsection{Synergy Quantification}

% Analytical quantification for small systems
%TODO
For trivial systems, we use (method)...

% Approximation for >3 size systems
% Maybe look at Kraskov how to do a kNN approach?
%TODO
For larger system, we do an estimation using (method)...

\subsection{Simulation Methodology}
%NA EXPERIMENTEN

% General idea
In our simulation, we consider the synergetic properties, memory, and resilience of gene regulatory networks.
In a biological system, there is no clear set of input- and output variables.
However, we do have a development of the system through time.
As such, we have chosen to consider the system $A_{t=0}$ as the input system, and system $A_{t=\delta t}$ as the output system.
Here, $\delta t$ is an step in time. % in arbitrary units or in seconds, once we start using rates?
The size of the set $A$ is in theory arbitrarily big, but in reality limited by computational complexity of the evaluation of the model, which is in sections of the implementation $O(2^{c \dot n})$.

% System definition
%TODO
We use a continuous model that describes a gene regulatory system.
The system is defined through a set of independent initial distributions, which represent reactant concentrations.
For the sake of simplicity, we have assumed these distributions to be Gaussian.
The progression of the system over time is defined through a set of ODEs.
This ODE system takes the form of (base this on the paper I get the dataset from)...
The set of flux rates can be trained, and is ultimately what defines synergy in the system.
The rate matrix is defined to be sparse, as biological networks tend to not be fully connected, nor strongly connected.
% System development over time
To evaluate the system development over time, we use the \texttt{dop853}-solver in SciPy \cite{scipy}.
This is an explicit runge-kutta method of order 8(5,3).

% Cost function
As a measure for the impact of a nudge we use the relative entropy, the Kullback-Leibler divergence $D_\mathrm{KL}(A_i^\prime \| A_i)$.
We utilize a k-Nearest Neighboor based approximation, as described by Wang \cite{wang2009divergence}.
% Some information on nudges (methods)
In order to nudge a variable, we add random Gaussian noise to this variable in the sample.
As our initial distribution is also Gaussian, this yields a new Gaussian distribution with a higher variance.
The memory between two states is defined as $\mathrm{I}(A^{t=0};A^{t = \delta t})$, and is estimated using k-Nearest Neighbor approximation for mutual information as described by Kraskov \cite{kraskov2004estimating}.
We use the NPEET library implementation for both these properties \cite{versteeg2013NPEET}.

%TODO
%De wiskunde bespreken waarom MI op een deterministisch systeem kan.
We can use the mutual information despite the fact that the ODE system is deterministic, because...
%TODO
%Defend if normalization is still necessary if I use the KL-divergence and the MI (I do need to somehow invert the MI, as this should be high, and the divergence small)
Normalization is still necessary in the cost function, and is achieved by...

% Generating a random GRN-like network
% Some networkx required
% Defines the sparsity
%TODO
We generate a random ODE system by first generating a network using the NetworkX library \cite{networkx}...

% Training of free parameters
% If I use/implement succesfully genetic algorithm: add appendix!
We expect that our cost function is highly irregular, with many local minima that are not on the Pareto boundary.
We found that the best performing optimizer was the Broyden–Fletcher–Goldfarb–Shanno (BFGS) algorithm, as implemented in the SciPy library \cite{fletcher2013practical, scipy}.
The training is performed on a single sample of the initial distribibution.
If we were to resample every training iteration, the cost landscape would be too prone to change to progress training.
To ensure that the result is not biased by the sample, we resample several times, each time continuing training where the previous iteration left off.

\subsection{Hypotheses}

We hypothesize that:

% Hypotheses (move to end introduction)
\begin{itemize}
\item There is significantly more synergy in a biological GRN motif versus a random GRN motif
\item An actual GRN scores significantly better in memory and single-nudge resilience than a random GRN
\item The is a stronger than linear decrease in resilience when increasing the number of variables nudged
\item An actual GRN is at the Pareto boundary of the memory/resilience cost function
\item Synergy is found at a low level in biological networks, the level of common network motifs 
\item Synergy is found at a low level in trained random GRNs
\end{itemize}

\end{document}