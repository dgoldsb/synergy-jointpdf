% Version: 0.0

\documentclass[../main.tex]{subfiles}

\begin{document}

\begin{abstract}

In this study, we compare the level of synergy and memory, as well as the sensitivity to various nudges in gene regulation opposed to a completely random system.
We model a system through a joint probability mass function, that evolves in discrete timesteps using a deterministic transition table. 
Random systems are initialized with a completely random transition table, whereas tables converted from a random gene regulatory network (GRN) are used for gene regulation systems.
Our hypothesis was that increased synergy might result in a lower nudge impact in biological networks.
It was observed that the sample space of GRN-based transition tables is a small subspace of all transition tables.
We found that random networks are significantly higher in synergy and memory, but that they also suffer a higher impact from nudges.
A strong positive correlation was observed between the system memory and the impact of a nudge, and it was also found that in some systems a higher synergy lead to a lower nudge impact.
Random systems were found to be well-balanced between synergy and redundancy at all levels, whereas gene regulation systems are more prone to contain excess redundancy or synergy at some scale.
We suggest that the type of relations in gene regulatory networks are less likely to lead to duplications in the next timesteps (a high memory) and yield a lower synergy, resulting in an on average lower nudge impact.
For future research, we suggest to investigate whether similar levels of synergy are found in real networks.
If not, this would suggest that our model for generating GRNs is not complex enough.

\end{abstract}

\end{document}

