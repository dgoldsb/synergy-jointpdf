% Version: final

\documentclass[../main.tex]{subfiles}

\begin{document}

\begin{abstract}
Polyadic relationships have been suggested as vital elements in natural system, but they have remained largely unquantified and ignored in system analyses.
In this study, we compare the level of synergy and memory with the sensitivity to various nudges in gene regulation, and how this differs between a biological random system and a uniform random system.
We model a system through a joint probability mass function, that evolves in discrete time steps using a deterministic transition table.
Random systems are initialized with a completely random transition table, whereas tables converted from a random gene regulatory network (GRN) are used for gene regulation systems.
It was observed that the sample space of GRN-based transition tables is a small subspace of all transition tables.
Our hypothesis is that increased synergy results in a lower nudge impact in biological random motifs.
We found that uniform random motifs are significantly higher in synergy and memory, but that they also suffer a higher impact from nudges.
A strong positive correlation was observed between the system memory and the impact of a nudge, but the synergy of a system was found to be uncorrelated with the system resilience.
Random systems were found to be well-balanced between synergy and redundancy at all levels, whereas gene regulation systems are more prone to contain excess redundancy or synergy at some scale.
We suggest that the type of relations in gene regulatory networks only let the system evolve to a few states, which results in a much lower memory as information is lost during time evolution, and an on average lower nudge impact.
For future research, we suggest to investigate whether similar levels of synergy are found in real networks.
If not, this would suggest that our model for generating GRNs is not complex enough.
From this study, we conclude that synergy does not have the resilience-increasing role in networks that has been hypothesized in prior research.
\end{abstract}
\end{document}

