% Version: 0.0

\documentclass[../main.tex]{subfiles}

\begin{document}

% Set the stage
% Something about complexity
In the analysis of complex system, it is helpful to have a quantification of how 'complex' a system is.
Ideally, this quantification allows for the distinction of regular systems, chaotic systems, and systems that show complex behavior.
Information has grown to be a staple tool in many fields that work with complex systems \cite{williams2010nonnegative}. % Referenced to later (a)
Originally, the primary used concepts where mutual information and entropy.
Not super useful.
Another concept is the Langton parameter, which can be used to find cellular automata's that exihibit complex behavior (separating from chaos and regularity) \cite{langton1990computation}.
Expansion of IT in later years with synergy.
Synergy is a better predictor than the Langton parameter \cite{9999QuaxChli}

% Give different IT ideas
Basis is MI and conditional MI
Two extensions originally.
Total correlation \cite{watanabe1960information}. Nothing on the structure, a single number.
Interaction information McGill \cite{mcgill1954multivariate}. Entropy is first order, MI second order, more lies beyond. Unfortunate is that this can be negative.
No general consensus on which yet.
Third proposal is the one we follow, a system of redundancy, synergy and entropy in multivariate cases \cite{williams2010nonnegative}.
Decomposition scales to larger system, but explodes in terms of number of possible 'fields' in Venn diagram.
Calculate.

% Level 1: entropy
Shannon's entropy \cite{shannon1949mathematical}.

% Level 2: mutual information
Shannon' MI \cite{shannon1949mathematical}.

% Level 3: synergy/redundancy
% Give meaning to what synergy is
\cite{williams2010nonnegative}.
Synergy (...).
Make clear what the difference is between redundancy (2+1) and MI (2).
The sum of positive and negative info, hard to distinguish when taking I(S;R1;R1)

% Piece about synergistic function
X-or is perfect synergistic.
Explain..
% Take it to continuous
Cycle is highly synergetic, the continuous version of the X-OR; explain the similarity

\end{document}