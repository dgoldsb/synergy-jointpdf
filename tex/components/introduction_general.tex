% Version: 0.0

\documentclass[../main.tex]{subfiles}

\begin{document}
% General context
%% Put synergy, a term from information theory, at home in the computational science mindset
In complex systems, elementary properties of many individuals lead to emergent behavior on a system-size scale.
We can view this emergence as a ’synergy’ of system elements together, which is a vital part of the dynamics of the system as a whole.
In information theory, synergy as a quantifiable property of a system is a relatively new idea, and can be seen as an addition to system properties like entropy and mutual information.
The method of quantification for systems larger than the trivial case of 2 random variables is amongst the open problems in the field, but multiple measurements have been proposed \cite{}.
Synergy and mutual information are closely related properties, as synergy can be seen as negative mutual information; it is additional information on a random variable that is given by a combination of other random variables, yet not a part of the information captured in any of the individual random variables.
As such, visualizations of the mutual information in a system of dependent random variables can sketch an image of the synergy in a system.
One method of visualization is through mutual information profiles.
Early versions used pairwise mutual information between all variables to create a profile of a system \cite{bar2013computationally}. 
The extended full mutual information profile, proposed by Quax et al., uses the mutual information between all possible subsets of input variables with an output variable \cite{quax2017quantifying}.
This is hypothesized that it would allow for the identification whether synergy is present at a low level, between two variables, or at a higher level, between groups of variables.

%% Link to gene regulatory networks/biology
%% Talk about pertubation resistance
Many natural systems are complex in their nature, such as gene regulatory networks, bird flocking patterns, and pattern formation in banks of bivalves \cite{}.
These networks are relatively simple in their building blocks, but show complicated patterns on a global scale.
In gene regulation, the expression of genes is spatially regulated from the conception of the organism onwards.
A possible angle to the question "why do we see  complexity in biological complex systems" is from the perspective of resilience and memory.
Biological systems require a level of memory, a relation between the current state of the system and previous states of the system. 
They also require a level of noise resilience, as biological systems tend to experience shocks from external sources \cite{peixoto2012emergence}.
Maximizing one is not always in the best interest of the other; noise resilience is maximized when the system automatically defaults to a hard-coded state, but this leaves no room for system memory.
A maximized system memory, on the other hand, will never forget noise, causing noise to never die out over time.
It has been hypothesized by Quax et al. that synergy increases the resilience of a system against pertubation in a single input variable \cite{quax2017quantifying}.
This means that synergy can be utilized to make a system resistant to nudges, while retaining the ability to memorize previous states.
We assume that the nudges that biological systems experience the most, and should be resilient against, are single-variable nudges.
If true, this would fit well with the proposal that synergy is used to provide protection against these disturbances.
After all, synergy operates at a multi-variate level, whereas single variable nudge operate at a single-variate level.
The realization of a middle way, that maximizes the combination of both resilience and memory, might be through synergy.
As such, complex synergistic relationships in the system could contribute to its functioning in a noisy environment.
This potential relationship between synergy and resilience makes it very interesting to determine the amount of synergy in a biological network, especially considering that the relationship between system complexity and resilience is one of the primary unanswered questions of ecology \cite{}.
This amount has, to our best knowledge, not been quantified or explained.

% In this study, we want to...
In this work we aim to examine the links between the complexity of a complex system its resilience, and the memory of the system.
In particular, we are interested in the role of synergy in these networks, which is used in this study as a quantification of the system complexity.
This problem is recurrent in many other disciplines, as principles from information theory are broadly applied.
As a first step, we aim to quantify the amount of synergy in a biological complex system.
This should allow us to put a meaning to how present synergy is in real-world systems.
We focus on gene regulation networks, as small, elementary motifs are readily available in these networks.
Resilience is a hot topic in gene regulation, has been done discrete without synergy using Boolean networks \cite{peixoto2012emergence}.
We expect that this discrete approximation is not sufficient, as it has been found that the same Boolean motif does not always have the same function \cite{ingram2006network}.
As such, we will use a continuous ODE-based model for our computations.
The small size helps both in the quantification of synergy, and in followup investigation.
Secondly, we test the hypothesis that a gene regulation network has more synergy than a random gene regulation network of a similar size. 
We investigate this through a simulation study in continuous space, where networks are approximated locally in time through an ODE system.
In addition, we also measure whether real-world motifs score better in terms of system memory and single-variate nudge resilience than random networks, and whether these networks are Pareto optimal in these two properties.
Thirdly, we want to test our assumption that biological networks should be resilient to single-variable nudges, but are not necessarily resilient to nudges in multiple variable at once.
We do so by examining the resilience of a real GRN when nudging an increasing number of variables.
Finally, we take a more detailed look at the level at which synergy occurs using full mutual information profiles.
With this, we provide insight into larger gene regulation motifs, as well as a case for the use of these profiles in the analysis of synergy in complex systems.
\end{document}