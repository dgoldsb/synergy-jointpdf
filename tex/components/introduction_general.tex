% Version: 0.0

\documentclass[../main.tex]{subfiles}

\begin{document}

% General context
%% Put synergy, a term from information theory, at home in the computational science mindset
In complex systems, elementary properties of many individuals lead to a large scale mergent behavior.
We can view this emergence as a form of of ’synergy’ together which is a vital part of the dynamics of the system as a whole. 
Synergy as a quantifiable property of a system is a relatively new idea in information theory, and can be seen as an addition to the system entropy and mutual information.
The method of quantification is amongst the open problems in the field, but multiple measurements have been proposed \cite{}.
Synergy and mutual information are closely related properties, as synergy can be seen as negative mutual information; it is additional information on a random variable that is given by a combination of other random variables, yet not a part of the information captured in any of the individual random variables.
As such, visualizations of the mutual information in a system of dependent random variables can sketch an image of the synergy in a system.
One method of visualization is through mutual information profiles.
Early versions used pairwise mutual information between all variables to create a profile of a system \cite{bar2013computationally}. 
The extended full mutual information profile, proposed by Quax et al., uses the mutual information between all possible subsets of input variables with an output variable \cite{quax2017quantifying}.
This is hypothesized that it would allow for the identification whether synergy is present at a low level, between two variables, or at a higher level, between groups of variables.

%% Link to gene regulatory networks/biology
%% Talk about pertubation resistance
Many natural systems are complex in their nature, such as gene regulatory networks \cite{}.
These networks are relatively simple in their building blocks, but show complicated patterns on a global scale.
In gene regulation, the expression of genes is spatially regulated from the conception of the organism onwards.
The presence of this complexity seems to suggest a level of synergy.
This amount has, to our best knowledge, not been quantified or explained.
A possible angle to the question "why do we expect synergy in biological complex systems" is from the perspective of resilience and memory.
This is a natural line of enquiry, as the relationship between system complexity and resilience is one of the primary unanswered questions of ecology \cite{}.
Biological systems require a level of memory, a relation between the current state of the system and previous states of the system. 
They also require a level of noise resilience, as biological systems tend to experience shocks from external sources \cite{peixoto2012emergence}.
Maximizing one is not always in the best interest of the other; noise resilience is maximized when the system automatically defaults to a hard-coded state, but this leaves no room for system memory.
A maximized system memory, on the other hand, will never forget noise, causing noise to never die out over time.
It has been hypothesized by Quax et al. that synergy increases the resilience of a system against pertubation in a single input variable \cite{quax2017quantifying}. 
This means that synergy can be utilized to make a system resistant to nudges, while retaining the ability to memorize previous states. 
The realization of a middle way, that maximizes the combination of both resilience and memory, might be through synergy.

% In this study, we want to...
In this work we aim to examine the links between the complexity of a complex system its resilience, and the memory of the system.
In particular, we are interested in the role of synergy in these networks, which is used in this study as a quantification of the system complexity.
This problem is recurrent in many other disciplines, such as systems biology.
As a first step, we aim to quantify the amount of synergy in a biological complex system.
This should allow us to put a meaning to how present synergy is in real-world systems.
We focus on gene regulation networks, as small, elementary motifs are readily available in these networks.
The small size helps both in the quantification of synergy, and in followup investigation.
Secondly, we test the hypothesis that a gene regulation motif has more synergy than a random motif of a similar size. 
We investigate this through a simulation study in continuous space, where motifs are approximated locally in time through an ODE system.
In addition, we also measure whether real-world motifs score better in terms of system memory and resilience than random networks, and whether these motifs are Pareto optimal in these two properties.
Finally, we take a more detailed look at the level at which synergy occurs using full mutual information profiles.
With this, we provide insight into larger gene regulation motifs, as well as a case for the use of these profiles in the analysis of synergy in complex systems.

\bibliographystyle{plain}
\def\biblio{}
\end{document}