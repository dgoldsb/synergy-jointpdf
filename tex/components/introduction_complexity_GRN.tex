% Version: 0.0

\documentclass[../main.tex]{subfiles}

\begin{document}

\subsubsection{Describing gene regulatory networks}
% Refer to some nice sources for more information on sea urchin and GRN, bolouri is nice for latter, former [2] of kuhn

% The state of network simulation
%% What is a gene regulatory network?
The developmental growth of complex animals is driven by the spatial and temporal activation of gene transcription to mRNA, and consequentially further gene products such as proteins.
This spatial and temporal sequence of states is determined in the genomic regulatory code of an animal \cite{bolouri2002modeling, kuhn2009monte}.
This code specifies gene regulatory networks (GRNs), which are networks of activation- and suppression relationships between genes.
Understanding the GRNs that drive processes in cells is key to understanding how a fertilized egg, a single cell, can grow out to an animal, a large and complex symbiosis of billions of cells.
It can also help us understand the mechanisms behind some human diseases, such as cancer \cite{qian2008inference}.

%% How do we describe gene regulatory networks
Constructing models that capture a GRN accurately is a complicated process.
It is experimentally difficult to measure kinetic parameteres associated with cellular processes in vivo \cite{bolouri2002modeling}.
The most basic way to represent a network is by a full Boolean network \cite{bolouri2002modeling}.
This network maps relationships between genes as instantaneous 'switches', allowing the production of a crude model without in-depth knowledge of reaction rates and delays in the real system.
Boolean networks are constructed through arrayed gene expressions assays, which cluster related genes, followed by a regulatory linkage analysis, which disruptsthe activity of a gene to observe the impact on downstream genes in the network \cite{bolouri2002modeling, wu2013high}.
Typically, they are only accurate approximations for small networks \cite{karlebach2008modelling}.
Probabilistic elements can be incorporated in a Boolean network \cite{schlitt2007current}.
This is typically done when empirical data is lacking, and there is uncertainty about the relationships between genes in the network \cite{karlebach2008modelling}.
An extension upon probabilistic Boolean networks is the petrinet, which functions by mimicking the buildup of transcription products over several timesteps \cite{karlebach2008modelling}.
Only when a 'bucket' fills, the down- or upregulation relationship is applied.

%ferrell2011modeling LIST OF MODELS, use this summary and links for dataset, oscillation easy to model in Boolean
Typically, it is ideal to describe a GRN as a mixture of Boolean logic and continuous rules \cite{bolouri2002modeling}.
This is for the reason that Boolean motifs can have vastly different functions based on kinetic properties \cite{ingram2006network}.
They work under strong assumptions, for instance that inhibition dominates over activation \cite{}.
Continuous rules are commonly captured in an ODE system or other algebraic formalism that describes the slow reactions involved in activators and surpressors binding to DNA, as well as transcription and translation \cite{ingram2006network}.
An ODE system is typically preferred, as this mimics the reaction dynamics in the cell more closely, but is also more difficult to construct that a simpler continuous model.
Fast reactions, such as protein-protein interactions, can still be modelled as Boolean switches in these models, as they occur at completely different timescale.
A continuous model can be constructed  from the Boolean counterpart, when additional research is done through the measurement of kinetic data, followed by verification to measure the correspondence of the network with reality \cite{bolouri2002modeling}.
It is preferable to use a non-linear ODE, as GRNs are non-linear in nature \cite{qian2008inference, tyson2003sniffers}.
A proposed improvement upon the common ODE models is that of a sparse additive ODE model, able to capture nonlinear relationships \cite{wu2014sparse}.
Other types of models exist, notably stochastic models, hidden Markov models and multi-values Boolean models, but these fall outside the scope of this research \cite{bolouri2002modeling, wu2014sparse}.

The parameters in these models can be notoriously difficult to fit to emperical data, as this requires the search through multi-dimensional space for an optimum fit \cite{bolouri2002modeling, kuhn2009monte}.
This challenge has been tackled with some success using, amongst others, Monte Carlo methods and genetic programming paired with Kalman filtering \cite{qian2008inference, kuhn2009monte}.
More elaborate statistical methods have been developed as well, such as the modified elastic net-method, LASSO-methods and the Bayesian best subset regression \cite{greenfield2013robust, wu2014sparse}.

\subsubsection{Information theory and GRNs}

An inquiry into GRNs using information theory has been made as well, primarily at the level of common network motifs \cite{zhang2012chaotic}.
Positive feedback loops have been found to function as switches and memory units, whereas negative feedback loops have been found to have a noise suppressing or oscilation-inducing function.
Studies have been done into the prevalence of chaotic behavior in GRNs \cite{zhang2012chaotic}.
These resulted in the conclusion that chaotic motifs of size $n\ge 3$ exist, but that they are uncommon in real networks, as they require competition between multiple feedback loops, at least one of which should be a negative feedback loop \cite{zhang2012chaotic}.
In real networks, this condition are scarcely met, although some GRNs do meet this condition, most notably the $n=4$ GRN that regulates the P53-system.

\subsubsection{Available GRN models}

There is a modest selection of GRN models available for further research.
These models are typically released in the SBML-format, an enriched xml-datatype.

One of the best captured and most researched GRNs is the endomesoderm GRN of the sea urchin \cite{bolouri2002modeling, kuhn2009monte}.
This network describes the activity of gene regulation in the early development of the sea urchin embryo, and is still in the proccess of being updated as new studies are done \cite{urchinmodel}.
An attemt has been made to build a full ODE model based on the Boolean abstraction of this GRN using Monte Carlo methods.
This was met with some success, as 65\% of the maximum possible correspondence with emperical data was achieved \cite{kuhn2009monte}.
In this reaction rate-type ODE model, the mRNA concentration linked to a gene (a measure for gene activity) is expressed as
%
\begin{equation}
\frac{dX}{dt} = (\frac{k_A \cdot A(t)}{c_A + A(t)} + \frac{k_B \cdot B(t)}{c_B + B(T)}) \cdot \frac{k_C \cdot C(t)}{c_C + C(T)} - k_\mathrm{deg} X(t)
\end{equation}
%
where $A$, $B$ and $C$ are protein concentrations, and the lower case letters denote kinetic constants.
Such a model of chemical master equations is popular, like sigmoidal and Michaelis-Menten models, as most limiting processes in gene regulation are chemical reactions \cite{aijo2009learning}.
The model contains of 54 genes, 140 variable species, 278 reactions and 287 parameters.

A model that mimics real reaction rates less to capture GRNs is a system of continuous non-linear differential equations.
This has been implemented by \cite{qian2008inference} as
%
\begin{equation}
\frac{dx_i}{dt} = f_i(x_1,...,x_n) + v_i
\end{equation}
%
where
%
\begin{equation}
f_i = \sum_{j=1}^{L_i} [(w_{ij}+ \mu_{ij})\Omega_{ij}(x_1,...,x_n)]
\end{equation}
%
and $\Omega(x_1,...,x_n)$ is a non-linear function, such as a sigmoid function.
However, no dataset was freely available from this study.

% Yeast model
% Skip the artificial NN for now, we are looking at motifs
Most models, both ODE and Boolean types, have been made of the yeast \textit{S. cerevisiae}, and more recently also \textit{S. pombe} \cite{ferrell2011modeling}. As a result, using the \textit{S. cerevisiae} network is encouraged for most purposes, as this network is well validated, up-to-date with current technologies and widely accepted.

% Looking at a subsection
Due to the large size of GRNs, it is often not possible to apply computational methods on the network as a whole.
To circumvent this limitation, we often look at motifs in the network, isolated from the rest.
This can be donne simply by removing all nodes, save the nodes of interest \cite{zhang2012chaotic}.
When using probability distributions instead of set initial conditions, it is possible to capture the influence of the rest of the network on the input variables by enforcing correlations between them, thus drawing from a joint PDF.

\end{document}