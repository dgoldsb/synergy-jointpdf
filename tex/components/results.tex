% Version: 0.0

\documentclass[../main.tex]{subfiles}

\begin{document}

\subsection{Assumption Checks}

% TSNE
The t-SNE plots do support our assumption that we are sampling only a subspace of all possible systems with our GRN-like random systems.
We show two t-SNE plots in Fig.~\ref{fig:TSNE}, for both a smaller and larger system\footnote{Additional plots can be found in the repository \url{github.com/dgoldsb/synergy-jointpdf}}.
We see a very clear division between the completely random systems (red) and the biologically inspired systems (blue).
The fact that here and there a random system is mixed in with the biologically inspired system further indicates that we are dealing with a small subspace; there is a small chance when sampling from the complete space to draw a system from the subspace.

\begin{figure}[H]
    \centering
    \begin{subfigure}[b]{0.4\textwidth}
        \includegraphics[width=\textwidth]{./../result_pandas/k=2_l=4/tsne2D.pdf}
        \caption{t-SNE for $k=2$ and $l=4$ ($n=900$)}
    \end{subfigure}
    \begin{subfigure}[b]{0.4\textwidth}
        \includegraphics[width=\textwidth]{./../result_pandas/k=4_l=4/tsne2D.pdf}
        \caption{t-SNE for $k=4$ and $l=4$ ($n=900$)}
    \end{subfigure}
    \caption{t-SNE plots for varying experiments}
    \label{fig:TSNE}
\end{figure}

% Cycles
We find that biologically inspired systems are much less likely to contain cycles than entirely random systems.
In Fig.~\ref{fig:cycles} we show a distribution of the maximum cycle length for a small and large system.
We see that in the smallest systems some small cycles are formed for the GRN-like model.
These cycles, however, disappear once the system size is increased.
We find that the number of system states $l$ does not have a noticable impact on the cycle distribution in biological systems; we see that cycles become much less likely when increasing the system size, but not when increasing the number of system states. % this is weird, because in both cases we increase the number of states
As there is always either a cycle or a sink, a maximum cycle length of 1 is the lowest attainable for a system.

\begin{figure}[H]
    \centering
    \begin{subfigure}[b]{0.4\textwidth}
        \includegraphics[width=\textwidth]{./../result_pandas/k=2_l=2/cycles.pdf}
        \caption{Distribution for $k=2$ and $l=2$ ($n=900$)}
    \end{subfigure}
    \begin{subfigure}[b]{0.4\textwidth}
        \includegraphics[width=\textwidth]{./../result_pandas/k=3_l=2/cycles.pdf}
        \caption{Distribution for $k=3$ and $l=2$ ($n=900$)}
    \end{subfigure}
    \caption{Histograms of the maximum cycle length distribution}
    \label{fig:cycles}
\end{figure}

% Finding flower motifs, skip for now

% Normality and all
In our parameter sweep we drew a large number of separate samples containting the synergy, memory, and resilience-properties of networks.
We found that most of these sample where not normally distributed.
Furthermore, those that did fit a normal distribution often contained outliers, and had wildly varying variances.
As such, we do not meet the assumptions for a t-test for comparing means, and for the Pearson test of correlation.
To handle the sample correctly we use non-parametric statistics in all further analysis, such as the Wilcoxon signed-rank test and the Spearson test of correlation. % these tests throw away the actual values, so outliers  are no problem either, they will simply get the highest/lowest rank

\subsection{Comparisons GRN and Random Systems}

We compared the level of synergy measured in random and biologically inspired systems for varying system sizes and number of states.
We found that completely random systems carry significantly more synergy than the biological systems in all cases.
These results are compiled in Table~\ref{synergy}.

\begin{figure}[h]
\label{synergy}
\begin{tabular}{|c|c|c|c|}
\hline
\diagbox{\# nodes }{\# states}  & 2.0 & 3.0 & 4.0\\
\hline
2.0 & 32478.0*** \cellcolor{yellow!20} & 25229.0*** \cellcolor{yellow!20} & 21135.0*** \cellcolor{yellow!20}\\
\hline
3.0 & 9135.0*** \cellcolor{yellow!20} & 2759.0*** \cellcolor{yellow!20} & 2066.0*** \cellcolor{yellow!20}\\
\hline
4.0 & 356.0*** \cellcolor{yellow!20} & 115.0*** \cellcolor{yellow!20} & 434.0*** \cellcolor{yellow!20}\\
\hline
5.0 & 11.0*** \cellcolor{yellow!20} & 0.0*** \cellcolor{yellow!20} & \\
\hline
\end{tabular}
\centering
\caption{Experiment synergy, Z-value and significance per experiment (* implies $p<0.05$, ** $p<0.005$, *** $p<0.0005$) with n=900. Green background implies higher mean in the biological network, yellow higher mean in the random network.}
\end{figure}

We similarly compared the level of synergy measured in random and biologically inspired systems for varying experiments.
We found that completely random systems carry significantly more memory than the biological systems in most cases. % this is to be expected, the memory is really really high
These results are compiled in Table~\ref{memory}.

\begin{figure}[h]
\label{memory}
\begin{tabular}{|c|c|c|c|}
\hline
\diagbox{\# nodes }{\# states}  & 2.0 & 3.0 & 4.0\\
\hline
2.0 & 40878.5*** \cellcolor{yellow!20} & 6442.0*** \cellcolor{yellow!20} & 2509.0*** \cellcolor{yellow!20}\\
\hline
3.0 & 44352.0* \cellcolor{yellow!20} & 10494.0*** \cellcolor{yellow!20} & 3890.0*** \cellcolor{yellow!20}\\
\hline
4.0 & 30656.0*** \cellcolor{yellow!20} & 3652.0*** \cellcolor{yellow!20} & 1011.0*** \cellcolor{yellow!20}\\
\hline
5.0 & 0.0*** \cellcolor{green!20} & 0.0*** \cellcolor{green!20} & \\
\hline
\end{tabular}
\centering
\caption{Experiment memory, Z-value and significance per experiment (* implies $p<0.05$, ** $p<0.005$, *** $p<0.0005$) with n=900. Green background implies higher mean in the biological network, yellow higher mean in the random network.}
\end{figure}

We did not find a consistent difference in the impact of nudging a single variable.
As shown in Table~\ref{resilience_single}, the nudge impact is slightly higher in some cases for biological networks.
However, in most experiments there was no significant difference, or a difference in the other direction.
Furthermore, all experiments with larger system sizes showed no difference.
These experiments are closer to reality, as gene regulatory networks consist of many genes.

\begin{figure}[h]
\label{resilience_single}
\begin{tabular}{|c|c|c|c|c|}
\hline
\# nodes & \diagbox{\# states}{$\epsilon$}  & 0.1 & 0.25 & 0.5\\
\hline
\multirow{3}{*}{2.0} & 2.0 & 5300.0 & 5399.0 & 5466.0\\
\cline{2-5}
  & 3.0 & 4257.0* \cellcolor{green!20} & 4691.0 & 4645.0\\
\cline{2-5}
  & 4.0 & 3357.0*** \cellcolor{green!20} & 3840.0*** \cellcolor{green!20} & 4312.0* \cellcolor{green!20}\\
\cline{2-5}
\hline
\multirow{3}{*}{3.0} & 2.0 & 5290.0 & 3745.0*** \cellcolor{yellow!20} & 5270.0\\
\cline{2-5}
  & 3.0 & 4145.0** \cellcolor{green!20} & 5241.0 & 5033.0\\
\cline{2-5}
  & 4.0 & 4469.0* \cellcolor{green!20} & 5450.0 & 5301.0\\
\cline{2-5}
\hline
\multirow{3}{*}{4.0} & 2.0 & 5316.0 & 4356.0* \cellcolor{yellow!20} & 4177.0* \cellcolor{yellow!20}\\
\cline{2-5}
  & 3.0 & 5489.0 & 5489.0 & 5503.0\\
\cline{2-5}
  & 4.0 & 5143.0 & 5546.0 & 5136.0\\
\cline{2-5}
\hline
\multirow{3}{*}{5.0} & 2.0 & 3102.0*** \cellcolor{yellow!20} & 5295.0 & 4480.0* \cellcolor{green!20}\\
\cline{2-5}
  & 3.0 & 3527.0*** \cellcolor{yellow!20} & 5458.0 & 5436.0\\
\cline{2-5}
  & 4.0 &  &  & \\
\cline{2-5}
\hline
\end{tabular}
\centering
\caption{Experiment resilience single, Z-value and significance per experiment (* implies $p<0.05$, ** $p<0.005$, *** $p<0.0005$) with n=300. Green background implies higher mean in the biological network, yellow higher mean in the random network.}
\end{figure}


We did find a consistent difference in the impact of nudging all variables.
As shown in Table~\ref{resilience_multiple}, the nudge impact is consistently higher in biological networks.
It appears that models with a higher resolution, by virtue of having more expression levels, show a stronger difference in the nudge impact.

\begin{figure}[h]
\label{resilience_multiple}
\begin{tabular}{|c|c|c|c|c|}
\hline
\# nodes & \diagbox{\# states}{$\epsilon$}  & 0.1 & 0.25 & 0.5\\
\hline
\multirow{3}{*}{2.0} & 2.0 & 5300.0 & 4726.0 & 4714.0\\
\cline{2-5}
  & 3.0 & 4095.0** \cellcolor{green!20} & 4389.0* \cellcolor{green!20} & 4152.0** \cellcolor{green!20}\\
\cline{2-5}
  & 4.0 & 3190.0*** \cellcolor{green!20} & 2864.0*** \cellcolor{green!20} & 3135.0*** \cellcolor{green!20}\\
\cline{2-5}
\hline
\multirow{3}{*}{3.0} & 2.0 & 5225.0 & 4490.0* \cellcolor{yellow!20} & 5658.0\\
\cline{2-5}
  & 3.0 & 3518.0*** \cellcolor{green!20} & 4816.0 & 5157.0\\
\cline{2-5}
  & 4.0 & 3741.0*** \cellcolor{green!20} & 3949.0** \cellcolor{green!20} & 3370.0*** \cellcolor{green!20}\\
\cline{2-5}
\hline
\multirow{3}{*}{4.0} & 2.0 & 5207.0 & 5448.0 & 5360.0\\
\cline{2-5}
  & 3.0 & 3618.0*** \cellcolor{green!20} & 4045.0** \cellcolor{green!20} & 4594.0* \cellcolor{green!20}\\
\cline{2-5}
  & 4.0 & 3653.0*** \cellcolor{green!20} & 3820.0*** \cellcolor{green!20} & 4349.0* \cellcolor{green!20}\\
\cline{2-5}
\hline
\multirow{3}{*}{5.0} & 2.0 & 397.0*** \cellcolor{yellow!20} & 2029.0*** \cellcolor{yellow!20} & 4156.0** \cellcolor{yellow!20}\\
\cline{2-5}
  & 3.0 & 577.0*** \cellcolor{yellow!20} & 2360.0*** \cellcolor{yellow!20} & 4743.0\\
\cline{2-5}
  & 4.0 &  &  & \\
\cline{2-5}
\hline
\end{tabular}
\centering
\caption{Experiment resilience multiple, Z-value and significance per experiment (* implies $p<0.05$, ** $p<0.005$, *** $p<0.0005$) with n=300. Green background implies higher mean in the biological network, yellow higher mean in the random network.}
\end{figure}

In Fig.~\ref{fig:3dscatter} we provide several scatterplots of experiments, with varying system sizes and numbers of expression levels.
We find that the variance reduces significantly when increasing the number of possible states, showing a much clearer pattern.
The same effect is seen in a lesser extend when increasing the system size.
This is likely due to the increase in the sample space; the sample space grows strongly when increasing either the system size or number of possible states.
We notice that random systems are very tightly clustered, and typically have an extremely high memory paired with high synergy.
The biological systems are much more spread out, and while there are outliers with higher memory than any random system, most samples have a lower memory and synergy.
The spread of nudge impacts is also much high in biological systems than in random systems.

\begin{figure}[H]
    \centering
    \begin{subfigure}[b]{0.45\textwidth}
        \includegraphics[width=\textwidth]{./../result_pandas/k=2_l=2_e=0.250000/scatter3D_memory_synergy_resilience.pdf}
        \caption{Distribution for $k=2$ and $l=2$ ($n=900$)}
    \end{subfigure}
    \begin{subfigure}[b]{0.45\textwidth}
        \includegraphics[width=\textwidth]{./../result_pandas/k=2_l=4_e=0.250000/scatter3D_memory_synergy_resilience.pdf}
        \caption{Distribution for $k=2$ and $l=4$ ($n=900$)}
    \end{subfigure}
\bigskip
    \begin{subfigure}[b]{0.45\textwidth}
        \includegraphics[width=\textwidth]{./../result_pandas/k=4_l=2_e=0.250000/scatter3D_memory_synergy_resilience.pdf}
        \caption{Distribution for $k=4$ and $l=2$ ($n=900$)}
    \end{subfigure}
    \begin{subfigure}[b]{0.45\textwidth}
        \includegraphics[width=\textwidth]{./../result_pandas/k=4_l=4_e=0.250000/scatter3D_memory_synergy_resilience.pdf}
        \caption{Distribution for $k=4$ and $l=4$ ($n=900$)}
    \end{subfigure}
    \caption{Scatterplots of synergy, memory and nudge impact (varying $l$ and $k$)}
    \label{fig:3dscatter}
\end{figure}

\subsection{Spearman Results}
% TODO: report min/max somewhere?

In our correlation experiments we find a strong correlation between the synergy and the memory in a system.
Interestingly, the correlation is reversed in sign when comparing random networks and biologically-inspired networks.
In Table~\ref{GRN_rho_syn_mem}, we find that there is a highly significant, mostly positive correlation between synergy and memory.
In the 5 system size experiment, only memory values of 1 where reported, indicating that something went wrong in this experimental setup. % TODO: look into code what happens
In Fig.~\ref{fig:3dscatter} we see that the range over which the observed synergy varies is very large for biological networks.
Combinations of low synergy and high memory never occur.
In Table~\ref{random_rho_syn_mem}, we find a highly significant and strong relationship between synergy and memory.
However, the random sample only covers the high synergy-range.
As such, the random networks also do not populate the low synergy-high memory space.

\begin{figure}[h]
\label{GRN_rho_syn_mem}
\begin{tabular}{|c|c|c|c|}
\hline
\diagbox{\# nodes }{\# states}  & 2.0 & 3.0 & 4.0\\
\hline
2.0 & 0.06 & -0.12*  & -0.22*** \\
\hline
3.0 & 0.62***  & 0.22***  & 0.27*** \\
\hline
4.0 & 0.67***  & 0.35***  & 0.42*** \\
\hline
5.0 & -0.02 & -0.07 & nan\\
\hline
\end{tabular}
\centering
\caption{Experiment GRN rho syn mem, $r_S$ and significance per experiment for GRN tables (* implies $p<0.05$, ** $p<0.005$, *** $p<0.0005$) with n=900.}
\end{figure}

\begin{figure}[h]
\label{random_rho_syn_mem}
\begin{tabular}{|c|c|c|c|}
\hline
\diagbox{\# nodes }{\# states}  & 2.0 & 3.0 & 4.0\\
\hline
2.0 & -0.03 & -0.28***  & -0.40*** \\
\hline
3.0 & -0.18***  & -0.56***  & -0.73*** \\
\hline
4.0 & -0.46***  & -0.67***  & -0.77*** \\
\hline
5.0 & -0.53***  & -0.73***  & nan\\
\hline
\end{tabular}
\centering
\caption{Experiment random rho syn mem, $r_S$ and significance per experiment for random tables (* implies $p<0.05$, ** $p<0.005$, *** $p<0.0005$) with n=900.}
\end{figure}

We do not find a significant relationship between synergy and the impact of a single target nudge (Table~\ref{GRN_rho_syn_singleimpact} and Table~\ref{random_rho_syn_singleimpact}).
This result is the same in both random networks and GRN networks.
We find that both sets of experiments only contain a few significant results.
However, these $p$-values are relatively high, as they fall in the range $0.05 > p > 0.005$.
As we do close to 40 experiments, we should be wary of false positives by chance; with this number of experiments, results with a $p$-value of $p < 0.001$ should be considered with care.
Furthermore, the found correlations are weak, implying that even if they are significant they might not be relevant. % TODO: move this to results?

\begin{figure}[h]
\label{GRN_rho_syn_singleimpact}
\begin{tabular}{|c|c|c|c|c|}
\hline
\# nodes & \diagbox{\# states}{$\epsilon$}  & 0.1 & 0.25 & 0.5\\
\hline
\multirow{3}{*}{2.0} & 2.0 & -0.05 & -0.01 & -0.17* \\
\cline{2-5}
  & 3.0 & -0.01 & -0.04 & -0.08\\
\cline{2-5}
  & 4.0 & -0.00 & 0.04 & 0.07\\
\cline{2-5}
\hline
\multirow{3}{*}{3.0} & 2.0 & 0.10 & -0.05 & -0.04\\
\cline{2-5}
  & 3.0 & -0.06 & -0.00 & 0.11\\
\cline{2-5}
  & 4.0 & 0.15 & -0.04 & -0.06\\
\cline{2-5}
\hline
\multirow{3}{*}{4.0} & 2.0 & -0.17*  & -0.02 & -0.02\\
\cline{2-5}
  & 3.0 & -0.04 & -0.05 & 0.12\\
\cline{2-5}
  & 4.0 & 0.02 & 0.03 & 0.10\\
\cline{2-5}
\hline
\multirow{3}{*}{5.0} & 2.0 & 0.06 & 0.05 & -0.17* \\
\cline{2-5}
  & 3.0 & 0.09 & -0.12 & -0.18* \\
\cline{2-5}
  & 4.0 & nan & nan & nan\\
\cline{2-5}
\hline
\end{tabular}
\centering
\caption{Experiment GRN rho syn singleimpact, $r_S$ and significance per experiment for GRN tables (* implies $p<0.05$, ** $p<0.005$, *** $p<0.0005$) with n=300.}
\end{figure}

\begin{figure}[h]
\label{random_rho_syn_singleimpact}
\begin{tabular}{|c|c|c|c|c|}
\hline
\# nodes & \diagbox{\# states}{$\epsilon$}  & 0.1 & 0.25 & 0.5\\
\hline
\multirow{3}{*}{2.0} & 2.0 & -0.05 & -0.01 & -0.17* \\
\cline{2-5}
  & 3.0 & -0.01 & -0.04 & -0.08\\
\cline{2-5}
  & 4.0 & -0.00 & 0.04 & 0.07\\
\cline{2-5}
\hline
\multirow{3}{*}{3.0} & 2.0 & 0.10 & -0.05 & -0.04\\
\cline{2-5}
  & 3.0 & -0.06 & -0.00 & 0.11\\
\cline{2-5}
  & 4.0 & 0.15 & -0.04 & -0.06\\
\cline{2-5}
\hline
\multirow{3}{*}{4.0} & 2.0 & -0.17*  & -0.02 & -0.02\\
\cline{2-5}
  & 3.0 & -0.04 & -0.05 & 0.12\\
\cline{2-5}
  & 4.0 & 0.02 & 0.03 & 0.10\\
\cline{2-5}
\hline
\multirow{3}{*}{5.0} & 2.0 & 0.06 & 0.05 & -0.17* \\
\cline{2-5}
  & 3.0 & 0.09 & -0.12 & -0.18* \\
\cline{2-5}
  & 4.0 & nan & nan & nan\\
\cline{2-5}
\hline
\end{tabular}
\centering
\caption{Experiment random rho syn singleimpact, $r_S$ and significance per experiment for random tables (* implies $p<0.05$, ** $p<0.005$, *** $p<0.0005$) with n=300.}
\end{figure}

No significant relationship synergy and single nudge impact (Table~\ref{GRN_rho_syn_multimpact} and Table~\ref{random_rho_syn_multimpact})
Only a few hits, and with a 0.05 > p-value > 0.005
We do 36 experiments, so this is not unexpected
Also not hugh correlations
We can wonder about the relevance % RESULTS
alfa lager,
significant en relevantie


We do not find a significant relationship between synergy and the impact of a nudge that targets all variables in the system (Table~\ref{GRN_rho_syn_singleimpact} and Table~\ref{random_rho_syn_singleimpact}).
This result is the same in both random networks and GRN networks.
Again, we find only several experiments with a $p$-value in the range $0.05 > p > 0.005$, and with a low correlation $r_S$.

\begin{figure}[h]
\label{GRN_rho_syn_multimpact}
\begin{tabular}{|c|c|c|c|c|}
\hline
\# nodes & \diagbox{\# states}{$\epsilon$}  & 0.1 & 0.25 & 0.5\\
\hline
\multirow{3}{*}{2.0} & 2.0 & -0.09 & -0.09 & -0.07\\
\cline{2-5}
  & 3.0 & 0.01 & 0.08 & 0.05\\
\cline{2-5}
  & 4.0 & -0.02 & 0.04 & 0.13\\
\cline{2-5}
\hline
\multirow{3}{*}{3.0} & 2.0 & 0.19*  & -0.07 & -0.06\\
\cline{2-5}
  & 3.0 & 0.04 & -0.07 & 0.19* \\
\cline{2-5}
  & 4.0 & 0.23*  & 0.07 & 0.04\\
\cline{2-5}
\hline
\multirow{3}{*}{4.0} & 2.0 & -0.06 & -0.02 & -0.22* \\
\cline{2-5}
  & 3.0 & 0.08 & 0.12 & 0.17* \\
\cline{2-5}
  & 4.0 & 0.10 & 0.05 & 0.02\\
\cline{2-5}
\hline
\multirow{3}{*}{5.0} & 2.0 & 0.11 & 0.07 & 0.02\\
\cline{2-5}
  & 3.0 & 0.10 & -0.01 & -0.09\\
\cline{2-5}
  & 4.0 & nan & nan & nan\\
\cline{2-5}
\hline
\end{tabular}
\centering
\caption{Experiment GRN rho syn multimpact, $r_S$ and significance per experiment for GRN tables (* implies $p<0.05$, ** $p<0.005$, *** $p<0.0005$) with n=300.}
\end{figure}

\begin{figure}[h]
\label{random_rho_syn_multimpact}
\begin{tabular}{|c|c|c|c|c|}
\hline
\# nodes & \diagbox{\# states}{$\epsilon$}  & 0.1 & 0.25 & 0.5\\
\hline
\multirow{3}{*}{2.0} & 2.0 & -0.09 & -0.09 & -0.07\\
\cline{2-5}
  & 3.0 & 0.01 & 0.08 & 0.05\\
\cline{2-5}
  & 4.0 & -0.02 & 0.04 & 0.13\\
\cline{2-5}
\hline
\multirow{3}{*}{3.0} & 2.0 & 0.19*  & -0.07 & -0.06\\
\cline{2-5}
  & 3.0 & 0.04 & -0.07 & 0.19* \\
\cline{2-5}
  & 4.0 & 0.23*  & 0.07 & 0.04\\
\cline{2-5}
\hline
\multirow{3}{*}{4.0} & 2.0 & -0.06 & -0.02 & -0.22* \\
\cline{2-5}
  & 3.0 & 0.08 & 0.12 & 0.17* \\
\cline{2-5}
  & 4.0 & 0.10 & 0.05 & 0.02\\
\cline{2-5}
\hline
\multirow{3}{*}{5.0} & 2.0 & 0.11 & 0.07 & 0.02\\
\cline{2-5}
  & 3.0 & 0.10 & -0.01 & -0.09\\
\cline{2-5}
  & 4.0 & nan & nan & nan\\
\cline{2-5}
\hline
\end{tabular}
\centering
\caption{Experiment random rho syn multimpact, $r_S$ and significance per experiment for random tables (* implies $p<0.05$, ** $p<0.005$, *** $p<0.0005$) with n=300.}
\end{figure}

We provide similar tables to Table~\ref{GRN_rho_syn_singleimpact}-~\ref{random_rho_syn_multimpact} in the appendix for memory instead of synergy.
The results, however, where found to be similar.
This is not unsurprising; synergy and memory where found to be highly correlated.
It appears from this result that the information in the random variable memory that is uncorrelated with the synergy is not correlated with nudge impact either.

% rest of plots in git release, footnote

\subsection{MI-Profile}

In the MI-profiles, we find that the synergy-redundancy ratio is less balanced in biology-like systems than in completely random systems.
Like in previous experiments, we find that increasing the number of possible expression levels $l$ makes the difference between the two system types more pronounced.
In Fig.~\ref{fig:profilel4}, we show the ensembles of MI-profiles for both random and GRN-like systems.
We find that random systems stay much closer to a straight line than GRN-like systems, implying that at all levels similar amounts of redundancy and synergy occur.
In biological systems, this balance is less present, as we see that in many samples there is much more redundancy oynergy at some levels.
This is less present, but still noticable, for fewer expression levels (Fig.~\ref{fig:profilel2}).

\begin{figure}[H]
    \centering
    \begin{subfigure}[b]{0.4\textwidth}
        \includegraphics[width=\textwidth]{./../result_pandas/k=3_l=2/MIprofile_random.pdf}
        \caption{Profile ensemble of random systems}
    \end{subfigure}
    \begin{subfigure}[b]{0.4\textwidth}
        \includegraphics[width=\textwidth]{./../result_pandas/k=3_l=2/MIprofile_GRN.pdf}
        \caption{Profile ensemble of biology-like systems}
    \end{subfigure}
    \caption{MI-profiles with $k=3$ and $l=2$ ($n=900$)}
    \label{fig:profilel2}
\end{figure}

\begin{figure}[H]
    \centering
    \begin{subfigure}[b]{0.4\textwidth}
        \includegraphics[width=\textwidth]{./../result_pandas/k=3_l=4/MIprofile_random.pdf}
        \caption{Profile ensemble of random systems}
    \end{subfigure}
    \begin{subfigure}[b]{0.4\textwidth}
        \includegraphics[width=\textwidth]{./../result_pandas/k=3_l=4/MIprofile_GRN.pdf}
        \caption{Profile ensemble of biology-like systems}
    \end{subfigure}
    \caption{MI-profiles with $k=3$ and $l=4$ ($n=900$)}
    \label{fig:profilel4}
\end{figure}

% I am leaving this out for now, the hypothesis was poorly phrased to begin with; the nudge impact cannot increase linearly with this type of nudgeand impact measurement
% we also already include the nudge width in a table, which proves enough of a point
%\begin{figure}[H]
%    \centering
%    \includegraphics[width=\textwidth]{./../result_pandas/k=3_l=4_e=0.250000/impacts.pdf}
%    \caption{Hidden layer output}
%    \label{fig:ugh}
%\end{figure}

%\subsection{Spread and Contrasts}
% drop this? not so interesting, we already take this from the 3d plot
% just make a note that this is all available for download


\end{document}