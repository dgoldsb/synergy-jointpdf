% Version: 0.0

\documentclass[../main.tex]{subfiles}

\begin{document}

% History of IT and biology
While ecology does deal with complex systems at many different scales, information theory never took hold as the primary analytical method \cite{ulanowicz2001information}. % Reference to earlier (A)
A large unanswered question inecology is the relationship between stability and complexity of ecosystems.
Conflicting answers to question of stability in complex or non-complex \cite{pimm1984complexity}.
The question was raised early, but no answer was found.
The evenness H, the entropy, was used to quantify complexity of a system.

On point of view was that with complexity, there were more pathways to reach a consumer.
Entropy over flows is used \cite{macarthur1955fluctuations},

In the past decade, computational studies have been done to answer the paradox. 
In general ecological theory, smaller more stable, then why so many big complex food networks \cite{kondoh2003foraging}?
One proposal is flexibility in structure of food webs.

Clearly, information theory has had a place.
However, we also see that we have moved away from the information theory complexity used from the 50s to 90s, to computational studies.
Two general approaches, according to \citep{ulanowicz2001information}.

% METE and Shannon entropy
On is based on Shannon entropy, and the quasi-static stock numbers.
This did not prove very useful to explain complexity.
This is used in METE marquet2014theory, more a distribution kind of thing, but with implications to stability.
elaborate mete, information theory application \cite{phillips2006maximum} % maybe throw out
Stock itself failed according to \citep{ulanowicz2001information}.
Because most ecologists think in stock, not flows, IT was written off.

% Flows approach
Second movement \cite{ulanowicz2009quantifying}.
Continue from MacArthur, reactionary against Shannon.
Applies H on flows, to contrast efficiency of pathways against robustness of a network.
Speaks of reserves, uses entropy-related measure t quantify.

% General biology link
% End how we might be able to answer it now
LINK THROUGH RESILIENCE, THIS IS THE QUESTION THAT ECOLOGISTS WANT ANSWERED IN THE END
Quax thinks \cite{quax2017quantifying}
Synergy might be interesting here
Shift away from the low-level IT applied here (entropy), and start looking at relationships between stochastic variables.
Don't treat a 'unit' as a random variable, which can be a specific species.
Instead, consider the number of each species a random variable, subject to noise from outside of the model.
As soon as we do that, we can start looking at interactions in the IT sense.
If we want to examine redundancy and synergy in a biological network, we can use the time dimension as a way to establish relationships
We can compare the system now with the system later
Useful measure is the halflife of the MI of a shock with the future state \cite{QuaxPersonal}
We cann see this is as memory
Simultaneously we need resilience
Can also be tested computationally with models
Synergy can be derived from these systems

\end{document}