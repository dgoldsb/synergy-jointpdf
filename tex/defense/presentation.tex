\documentclass[hyperref={pdfpagelabels=false}]{beamer}
\usepackage{lmodern}
\usetheme{Frankfurt}
\usepackage{natbib}
\usepackage{apalike}
\usepackage{amsmath}
\usepackage{multirow}
\usepackage{graphicx}
\usepackage{listings}
\usepackage{caption}
\usepackage{subcaption}
\usepackage{diagbox}
\usepackage{colortbl}
\usepackage{graphicx}
\usepackage{tikz}
\usetikzlibrary{arrows,shapes,snakes,automata,backgrounds,petri}
\usepackage{geometry}
\lstset{basicstyle=\ttfamily, escapeinside={\%*}{*)}}

\title{The role of synergy in the memory and resilience of random discrete gene regulatory networks}
\author{Dylan Goldsborough}
\date{12\textsuperscript{th} of March 2017}

% Citations
\bibliographystyle{plain}

\begin{document}
\begin{frame}
\titlepage
\end{frame} 

%% Structure
\begin{frame}
\frametitle{Table of contents}
\tableofcontents
\end{frame} 

%% LOOP DOOR MIJN THESIS HEEN EN VIND DE JUISTE ZINNEN MET REFERENCES

%%% Introduction
\section{Introduction} 
\setcounter{subsection}{1}

% STORY: I have always found this fascinating, how a complicated system can both attribuate to stability and chaos
% I should tell why absence and excess both change this
%
\begin{frame}
\frametitle{Introduction}
\begin{itemize}
\item Many natural systems are complex
\begin{itemize}
\item Consist of many units
\item Many interactions between units
\item Emergent properties at a higher level
\end{itemize}
\item Both absence and excess of complexity make systems more vulnerable to change \cite{kondoh2003foraging, macarthur1955fluctuations} \dots{}
\begin{itemize}
\item Less importance of a single interaction
\item Vulnerability of large systems to cascades
\end{itemize}
\item What is the relationship between complicatedness and resilience in biological complex systems? 
\end{itemize}
\end{frame}

% STORY: Rick did research into synergy, found that it helps resilience
% Natural systems can only remain in existence if they can cope with the perturbations
\begin{frame}
\frametitle{Introduction}
\begin{itemize}
\item It has been suggested that synergy increases the resilience of a system against perturbation in a single input variable \cite{quax2017quantifying}
\item Natural systems require two things to operate \dots{}
\begin{itemize}
\item Resilience against single-variable perturbations
\item Retention of information over time ("memory")
\end{itemize}
\item Extreme memory brings low resilience, and vice versa
\item To function well, systems should optimize a combination of memory and resilience
\item \textbf{Idea}: what if synergy plays a role in optimizing this balance?
\end{itemize}
\end{frame}

\begin{frame}
\frametitle{Introduction}
\begin{itemize}
\item Focus so far on dyadic interactions of variables \cite{ideker2001integrated, lu2004gene, tononi1999measures}
\begin{itemize}
\item Mutual information between random variables
\item Correlations between gene expression
\end{itemize}
\item Synergy is a polyadic relationship
\item Allows us to investigate properties that emerge at the polyadic level
\item Maybe this is a piece of the puzzle of the relationship between complexity, resilience and memory
\end{itemize}
\end{frame}

\begin{frame}
\frametitle{Introduction}
Partial information decomposition shows these higher level interactions.
Let us consider a system with random variables $X$ and $Y$ predicting $Z$.
\begin{itemize}
\item Unique information, contained only in $X$ or $Y$
\item Redundant information contained in both $X$ and $Y$
\item Synergetic information contained in neither $X$ and $Y$, but derived from a combination of the two
\end{itemize}
\end{frame}


\begin{frame}
\frametitle{Partial information-diagram showing a PID of a 3-variable system}
%% PICTURE %%
\def\firstcircle{(0:-0.9cm) circle (2cm)}
\def\secondcircle{(0:0cm) circle (3cm)}
\def\thirdcircle{(0:0.9cm) circle (2cm)}
\begin{figure}[ht]
\begin{center}
\begin{tikzpicture}
    \draw \firstcircle;
    \draw \secondcircle;
    \draw \thirdcircle;
   
    \begin{scope}[fill opacity=0.5]
        \clip \firstcircle;
        \fill[orange] \thirdcircle;
    \end{scope}
   
    \begin{scope}[even odd rule, fill opacity=0.5]
        \clip \thirdcircle (-3,-3) rectangle (3,3);
        \fill[yellow] \firstcircle;
    \end{scope}
   
    \begin{scope}[even odd rule, fill opacity=0.5]
        \clip \firstcircle (-3,-3) rectangle (3,3);
        \fill[red] \thirdcircle;
    \end{scope}
   
    \begin{scope}[even odd rule, fill opacity=0.3]
        \clip \firstcircle (-4,-4) rectangle (4,4);
        \clip \thirdcircle (-4,-4) rectangle (4,4);
        \fill[blue] \secondcircle;
    \end{scope}
   
    \node (x) at (-2,0)  {$\mathrm{I}\left(Z; X \right)$};
    \node (y) at (2,0)   {$\mathrm{I}\left(Z; Y \right)$};
    \node (r) at (0,0)   {$\mathrm{I}_\mathrm{red}\left( Z;X,Y \right)$};
    \node (s) at (0,2.3) {$\mathrm{I}_\mathrm{syn}\left( Z;X,Y \right)$};
    \node (w) at (0,3.2) {$\mathrm{I}\left( Z;X,Y \right)$};
\end{tikzpicture}
\end{center}
\label{venn}
\end{figure}
\end{frame}

% STORY: give an intuitive example of resilience and 
\begin{frame}
\frametitle{Introduction}
\begin{itemize}
\item An example of synergy is an XOR-gate
\item Quantifying synergy is an open problem \cite{griffith2011quantifying, olbrich2015information} \dots{}
\end{itemize}
\begin{table}[ht]
\begin{center}
\begin{tabular}{|c|c||c|}
\hline
$X$ & $Y$ & $Z$ \\
\hline
\hline
1 & 1 & 0 \\
1 & 0 & 1 \\
0 & 1 & 1 \\
0 & 0 & 0 \\
\hline
\end{tabular}
\end{center}
\caption{Truth table of an X-OR gate}
\label{XOR}
\end{table}
\end{frame}

% To set the expectations, kind of like the abstract
% Mention that the last two are double, we test it for both but found no difference
\begin{frame}
\frametitle{Research question}
\begin{itemize}
\item Is there a form of synergistic control on memory and resilience in gene regulatory networks?
\item Comparison of BRM and URM
\item Hypotheses \dots{}
\begin{itemize}
\item There is significantly more synergy in a BRM than in a URM
\item A BRM scores better in memory than a URM
\item A BRM scores better in single-nudge resilience than a URM
\item There is a positive correlation between synergy and resilience
\item There is a positive correlation between synergy and memory
\end{itemize}
\end{itemize}
\end{frame}

\section{Methodology}
\setcounter{subsection}{1}

%TODO continuous was too expensive, mention this
\begin{frame}
\frametitle{A model for GRNs}
\begin{itemize}
\item We model GRN motifs as discrete networks
\item A total of $n$ genes
\item Gene can be in state $m \in \{0, 1, ..., l\}$
\item Time is discrete
\item Deterministic model
\item Two representations in Python-implementation
\begin{itemize}
\item Graph form
\item Transition table form
\end{itemize}
\end{itemize}
\end{frame}

\begin{frame}
\frametitle{Graph form}
\begin{itemize}
\item Graph form mimics real GRN motifs
\item Genes can have four types of relations
\begin{itemize}
\item Stimulation (+)
\item Inhibition (-)
\item AND-stimulation (mimics co-factor)
\item AND-inhibition (mimics co-factor)
\end{itemize}
\item Non-stimulated genes experience decay over time
\item Self-stimulation is allowed \cite{thomas1995dynamical, zhou2016relative}
\end{itemize}
\end{frame}

\begin{frame}
\frametitle{Graph form}
PLAATJE NETWERK
\end{frame}

\begin{frame}
\frametitle{Transition table form}
\begin{itemize}
\item We can convert a graph form to a transition table
\item The transition table defines the state of the system at $t + \delta t$ for each possible system state at $t$
\item Total of $l^n$ states
\end{itemize}
\end{frame}

\begin{frame}
\frametitle{Transition table form}
\begin{table}[ht]
\begin{center}
\begin{tabular}{|c|c|c||c|c|c|}
\hline
$A_t$ & $B_t$ & $C_t$ & $A_{t+\delta t}$ & $B_{t+\delta t}$ & $C_{t+\delta t}$ \\
\hline
\hline
1 & 1 & 1 & 0 & 0 & 0 \\
1 & 1 & 0 & 1 & 0 & 1 \\
1 & 0 & 1 & 1 & 1 & 1 \\
1 & 0 & 0 & 0 & 0 & 0 \\
0 & 1 & 1 & 0 & 0 & 1 \\
0 & 1 & 0 & 1 & 0 & 1 \\
0 & 0 & 1 & 0 & 1 & 0 \\
0 & 0 & 0 & 1 & 1 & 1 \\
\hline
\end{tabular}
\end{center}
\caption{Transition table form for $n=3$ and $l=2$}
\end{table}
\end{frame}

\begin{frame}
\frametitle{Generating BRMs}
\begin{itemize}
\item We sample biological random motifs (BRMs) using the Erdős–Rényi algorithm \cite{margolin2006aracne}
\item We run the algorithm twice; for 1-to-1 edges, and for 2-to-1 edges
\item We allow self-referring edges
\item A random eligible function is attached to an edge
\item We then convert to a transition table to perform time evolution
\end{itemize}
\end{frame}

\begin{frame}
\frametitle{Generating URMs}
\begin{itemize}
\item We sample uniform random motifs (URMs) using a randomized design
\item We generate a transition table, skipping the graph form
\item Each expression level in the future state is randomly determined
\item We draw from a uniform distribution between $0$ and $l-1$
\end{itemize}
\end{frame}

\begin{frame}
\frametitle{Generating URMs}
\begin{table}[ht]
\begin{center}
\begin{tabular}{|c|c|c||c|c|c|}
\hline
$A_t$ & $B_t$ & $C_t$ & $A_{t+\delta t}$ & $B_{t+\delta t}$ & $C_{t+\delta t}$ \\
\hline
\hline
1 & 1 & 1 & ? & ? & ? \\
1 & 1 & 0 & ? & ? & ? \\
1 & 0 & 1 & ? & ? & ? \\
1 & 0 & 0 & ? & ? & ? \\
0 & 1 & 1 & ? & ? & ? \\
0 & 1 & 0 & ? & ? & ? \\
0 & 0 & 1 & ? & ? & ? \\
0 & 0 & 0 & ? & ? & ? \\
\hline
\end{tabular}
\end{center}
\caption{Transition table form for $n=3$ and $l=2$}
\end{table}
\end{frame}

\begin{frame}
\frametitle{Measurement methods: synergy}
\begin{itemize}
\item Most synergy measures are too expensive to compute
\item We use the mean of an upper- and lower bound
\item Upper: $\mathrm{I}\left( \mathbf{X}_{t=0}; \mathbf{X}_{t=\Delta t} \right) - \max_i [\mathrm{I}\left( X_{t=0,i};\mathbf{X}_{t=\Delta t}\right)]$
\item Lower: $\mathrm{WMS} \left( X;Y \right) = \mathrm{I} \left( X;Y \right) - \sum_i [\mathrm{I} \left( X_i;Y \right)]$
\item We normalize this to fall between 0 and 1 by dividing by $\mathrm{I}\left( \mathbf{X}_t ; \mathbf{X}_{t + \Delta t}\right)$
\end{itemize}
\end{frame}

\begin{frame}
\frametitle{Measurement methods: memory}
\begin{itemize}
\item We define the memory as the MI between the state at $t$ and $t + \delta t$
\item Expression: $\mathrm{I}\left(\mathbf{X}_t ; \mathbf{X}_{t + \Delta t}\right)$
\item We normalize by dividing by the entropy $\mathrm{H}\left(\mathbf{X}_{t + \Delta t}\right)$
\end{itemize}
\end{frame}

\begin{frame}
\frametitle{Measurement methods}
\begin{itemize}
\item We perform a local nudge on between 1 and $n$ genes
\item Joint PDF of the variables that are not nudged remains unaltered
\item Nudge size $0 \le \epsilon \le 1$, represents the fraction of the total probability tl be moved as part of the nudge
\item We define the resilience as the difference between the nudged and unnudged state at $t + \delta t$
\item Difference is defined using the Hellinger distance
\end{itemize}
\end{frame}

\begin{frame}
\frametitle{Complexity profile}
%
\begin{equation}
C_\mathrm{mult}(k) = \frac{1}{\binom{n}{k}}\frac{\sum_{X_i \in [\mathbf{X}]^k} [\mathrm{I}\left( X_i;Y \right)]}{\mathrm{I}\left( \mathbf{X};Y\right)}
\end{equation}
VOORBEELD COMPLEXITY PROFILE
\end{frame}

\begin{frame}
\frametitle{Experimental design}
\begin{itemize}
\item We draw a sample of BRMs and a sample of URMs
\item We compare the means (synergy, memory, resilience) using Wilcoxon rank test
\item We compute Spearman correlations within the samples
\begin{itemize}
\item Synergy and memory
\item Synergy and resilience
\item Memory and resilience
\end{itemize}
\item We generate complexity profiles
\end{itemize}
\end{frame}

\begin{frame}
\frametitle{Parameters}
\begin{itemize}
\item We do a parameter sweep
\item Average indegree of 4 \cite{lahdesmaki2003learning}
\item 75\% of edges are 1-to-1 
\end{itemize}
\begin{table}[H]
\begin{tabular}{| l | c | c | c |}
\hline
Parameter & Start & End & Increment \\
\hline
Network size (\#) & 2 & 5 & 1 \\
Logic size (\#) & 2 & 5 & 1 \\
Nudge size (fraction of probability) & 0.1 & 0.4 & 0.15 \\
\hline
\end{tabular}
\centering
\caption{The parameter ranges used for the experiments}
\label{parameters}
\end{table}
\end{frame}

\section{Results}
\setcounter{subsection}{1}
% STORY: analyse the results already in this section, I will only give a summary later

\begin{frame}
\frametitle{t-SNE}
\begin{figure}[ht]
    \centering
    \begin{subfigure}[b]{0.48\textwidth}
        \includegraphics[width=\textwidth]{../../result/k=2_l=2/tsne2D.pdf}
        \caption{t-SNE for $k=2$ and $l=4$ ($n=300$)}
    \end{subfigure}
    \begin{subfigure}[b]{0.48\textwidth}
        \includegraphics[width=\textwidth]{../../result/k=5_l=4/tsne2D.pdf}
        \caption{t-SNE for $k=5$ and $l=4$ ($n=300$)}
    \end{subfigure}
    \caption{t-SNE plots for varying experiments}
    \label{fig:TSNE}
\end{figure}
\end{frame}

\begin{frame}
\frametitle{Scatterplots}
\begin{figure}[ht]
    \centering
    \begin{subfigure}[b]{0.48\textwidth}
        \includegraphics[width=\textwidth]{../../result/k=3_l=2_e=0.250000/scatter3D_memory_synergy_resilience.pdf}
        \caption{Distribution for $k=3$ and $l=2$ ($n=300$)}
    \end{subfigure}
    \begin{subfigure}[b]{0.48\textwidth}
        \includegraphics[width=\textwidth]{../../result/k=5_l=4_e=0.250000/scatter3D_memory_synergy_resilience.pdf}
        \caption{Distribution for $k=5$ and $l=4$ ($n=300$)}
    \end{subfigure}
    \caption{Scatterplots of synergy, memory and nudge impact (varying $l$ and $k$, $\epsilon = 0.25$)}
    \label{fig:3dscatter}
\end{figure}
\end{frame}

\begin{frame}
\frametitle{Wilcoxon rank test}
\begin{itemize}
\item In all experiments URMs have a higher synergy
\item In all experiments URMs have a higher memory
\item In all experiments BRMs are more resilient
\item Significance increase with larger $n$ and $l$
\item Exeriments have $N = 100$
\end{itemize}
\end{frame}

\begin{frame}
\frametitle{Spearman correlation}
\begin{itemize}
\item Negative correlation between synergy and memory
\begin{itemize}
\item Not in all experiments observed, but majority
\item Relationship gets stronger with larger $n$ and $l$
\end{itemize}
\item Negative correlation between memory and resilience
\begin{itemize}
\item \textit{This relationship persists when we control for synergy}
\item Strong and highly significant in all experiments
\end{itemize}
\item Weak positive correlation between synergy and resilience
\begin{itemize}
\item \textit{This relationship disappears when we control for memory}
\end{itemize}
\item Correlations were observed in both URMs and BRMs
\end{itemize}
\end{frame}

\begin{frame}
\frametitle{Complexity profiles}
\begin{figure}[ht]
    \centering
    \begin{subfigure}[b]{0.4\textwidth}
        \includegraphics[width=\textwidth]{../../result/k=4_l=4/MIprofile_random.pdf}
        \caption{Profile ensemble of URMs}
    \end{subfigure}
    \begin{subfigure}[b]{0.4\textwidth}
        \includegraphics[width=\textwidth]{../../result/k=4_l=4/MIprofile_GRN.pdf}
        \caption{Profile ensemble of BRMs}
    \end{subfigure}
    \caption{MI-profiles with $k=4$ and $l=4$ ($n=900$)}
    \label{fig:profilel4}
\end{figure}
\end{frame}

\section{Conclusion}
\setcounter{subsection}{1}

\begin{frame}
\frametitle{Conclusion}
About what this means
Summary slide, discuss for a long time
Discuss Rick his hypothesis, and how I claim against this
\end{frame}

\begin{frame}
\frametitle{Conclusion}
Conclusion plaatje over mijn idee over Rick zijn ding.
No difference between URM BRM
No difference detected single multiple nudge
\end{frame}

\section{Discussion}
\setcounter{subsection}{1}

\begin{frame}
\frametitle{Limitations in experimental design}
* Een goed voorbeeld van het niveau waarop emergence/synergy gaat werken is flocking behavior; 4 vogels zullen zich nog niet als een flock gedragen, 100 wel. In het geval van genen is 5 genen niet genoeg, maar 10 motieven misschien wel.
\end{frame}

\begin{frame}
\frametitle{Limitations in the model}
* Stochastic in twee opzichten een verbetering: introduceer rest netwerk elke tijdstap, en stochastisch decay
\end{frame}

\section{Future research}
\setcounter{subsection}{1}

\begin{frame}
\frametitle{Future research}
Resources: repository
\end{frame}

\begin{frame}[allowframebreaks]
\frametitle{Literature}

\bibliography{../sources}
\end{frame}

%% BACKUP SLIDES

\section{Backup} 
\setcounter{subsection}{1}

% STORY: give an intuitive example of resilience and 
\begin{frame}
\frametitle{Introduction}
\begin{itemize}
\item Memory can be demonstrated with repeated die-rolls $X_t$
\begin{itemize}
\item Memory is maximized if we roll a die, and assume $X_{t=0} = X_{t=1}$
\item Memory is minimized if we roll a die, and do an independent reroll
\end{itemize}
\item Resilience can be seen as an adjustment of $X_{t=0}$
\begin{itemize}
\item Resilience is low when we roll $X_{t=0}$, adjust this, and then assume $X_{t=0} = X_{t=1}$
\item Resilience is higher when we roll several dice, adjust one, and reroll the others
\end{itemize}
\end{itemize}
\end{frame}

\end{document}
