\documentclass[11pt]{article}
\usepackage{fullpage}
\usepackage{setspace}
\pagestyle{empty}
\setlength{\tabcolsep}{0in}
\hyphenchar\font=-1
\usepackage{geometry}
\geometry{a4paper}
\setstretch{1}
\usepackage{booktabs}
\usepackage{topcapt}
\usepackage{tabulary}
\usepackage{hyphenat}
\usepackage{url}

\begin{document}
\noindent
\textbf{Start date}: February 6th, 2017\\
\textbf{End date}: August 6th, 2017\\
\textbf{Student}: Dylan Goldsborough (\url{dylan.goldsborough@student.uva.nl})\\
\textbf{Supervisor}: Rick Quax (\url{r.quax@uva.nl})\\
\textbf{Second reader}: Jaap Kaandorp (\url{j.kaandorp@uva.nl})\\

\section{Research proposition}

(Based on papers introduction to synergy.)
The question on relation synergy and resilience holds lot of practical relevance
Ecology major unanswered question
Also in neuropsychology, biological neural networks
Also ANN (Derk Jan).
Ultimately this can give us a better understanding of why systems are resilient, and a mean to predict the resilience of a system based on its composition alone.

\section{The main problem setting, research questions and/or hypothesis}

> Main problem: determinging links between the complexity of a complex system as a sum of its components, and its resilience to nudges. 
> RQ: is there a relation between the synergy profile as quantified following Rick Quax (\url{https://bitbucket.org/rquax/jointpdf}).
Hyp??
> Additional research question: does the relationship in simulated systems hold for real systems?
Hyp??

\section{Research design and method(s)}

Simulation study
> Extend Python library for continuous (Fourier)
> Generate random system both discrete and continuous
> Create synergy profiles and determine resilience
> Record

Applied section
> Apply to real world system
> Create synergy profiles and determine resilience
> Compare to what we know

\section{Project time table}

The project corresponds to a 42 EC studyload, and will officially be performed over the course of six months. 
The project will start on the 6th of February 2017, and end on the 6th of August in the same year.
The project contains three distinct stages: a literature study, a period of hypothesis testing, and an application to a real complex system. 

For the literature study, a total of 12 EC or 6 weeks has been reserved. 
There are two goals in this timeframe: (a) to describe the current state of research on quantifying synergy, complex system resilience, and relations between the two, and (b) selecting a real-life complex system of which to create a synergy profile, and to see if we can use this to explain the observed resilience. 
The preference in (b) lies in an ecological system of which the resilience is roughly known. 
The challenge lies in finding a suitable dataset, and determining how to interpret the system in the context of this synergy quantification.

Second, we will take to test our hypotheses in a simulation study. 
We will systematically generate random systems, with varying degrees of resilience, and examine if there is a relationship with the synergy profile. 
We will consider both systems the work with discrete probability functions, and continuous functions. 
The latter is a new development, which is necesarry for application to real-life systems.

Finally, we will take time to apply the same methods of the previous stage to a real complex system. 
The aim is here to compare our findings to what we know from emperical studies on these systems. 
For instance, some marine ecosystems have been found to not be resilient to heavy fishing activities on cod.
It would be interesting to see if we, based on the synergy profile of this system, would arrive at the same conclusion.
If we do, this would be a first step in answering a major unanswered question in ecology: is a highly complex ecosystem more resilient to nudges than a simple one, and why?
\end{document}
